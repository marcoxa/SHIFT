% -*- Mode: LaTeX -*-

% model.tex --



% Copyright (c)1996, 1997, 1998 The Regents of the University of
% California (Regents). All Rights Reserved. 
%
% Permission to use, copy, modify, and distribute this software and its
% documentation for educational, research, and not-for-profit purposes,
% without fee and without a signed licensing agreement, is hereby
% granted, provided that the above copyright notice, this paragraph and
% the following two paragraphs appear in all copies, modifications, and
% distributions. 
%
% Contact The Office of Technology Licensing, UC Berkeley, 2150 Shattuck
% Avenue, Suite 510, Berkeley, CA 94720-1620, (510) 643-7201, for
% commercial licensing opportunities. 
%
% IN NO EVENT SHALL REGENTS BE LIABLE TO ANY PARTY FOR DIRECT, INDIRECT,
% SPECIAL, INCIDENTAL, OR CONSEQUENTIAL DAMAGES, INCLUDING LOST PROFITS,
% ARISING OUT OF THE USE OF THIS SOFTWARE AND ITS DOCUMENTATION, EVEN IF
% REGENTS HAS BEEN ADVISED OF THE POSSIBILITY OF SUCH DAMAGE. 
%
% REGENTS SPECIFICALLY DISCLAIMS ANY WARRANTIES, INCLUDING, BUT NOT
% LIMITED TO, THE IMPLIED WARRANTIES OF MERCHANTABILITY AND FITNESS FOR
% A PARTICULAR PURPOSE.  THE SOFTWARE AND ACCOMPANYING DOCUMENTATION, IF
% ANY, PROVIDED HEREUNDER IS PROVIDED "AS IS". REGENTS HAS NO OBLIGATION
% TO PROVIDE MAINTENANCE, SUPPORT, UPDATES, ENHANCEMENTS, OR
% MODIFICATIONS. 


\section{The Model\label{model}}

\subsection{Components and Worlds}

A {\em component} is a part of a system, or a {\em world}.  The
behavior of a component (its time evolution) depends on its
{\em state}, and its environment (the visible part of other
components).

A {\em world} is an evolving set of components.  During the evolution
of the world new components may be created, and the manner
in which they interact may change.  At all times the behavior of the
world is self-contained, that is, it depends only on the state of its
components.

\subsection{Component Model}

A component is defined by its {\em state}, {\em inputs}, {\em
outputs}, and {\em exported events}; its {\em continuous time
evolution}; and its {\em discrete event evolution}.

At time $t$, the component's state $s(t)$ is the tuple $(x(t), q(t), l(t))$.
The vector $x(t) \in {\bf R}^n$ is the continuous state.  The discrete
state $q(t)$ is a state in a finite state machine.  The {\em link
state} $l(t)$ is a vector of references to other components.

Links describe both physical (e.g., wires and mechanical actuators)
and logical (e.g., sensors and communication channels)
interconnections between components.  Links are dynamic: the
components they reference may change in time.

A component's inputs, outputs, and exported events define the
component's interface to the rest of the world.  Outputs are variables
whose values are accessible (for reading) by other components.
Exported events are state machine transitions which can be
synchronized to those of other components.  Inputs are variables whose
values (during both continuous and discrete phases) are defined
externally to the component.

The continuous time evolution of the continuous state $x$ is defined
by differential equations and algebraic expressions.  The differential
equations are in the form \[ \dot{x}_i = f_i^{q(t)}(x(t), u(t), w(t))
\] where $u(t)$ is the input of the component, and $w(t)$ the output
of linked components.  The algebraic definitions are in the form \[
x_j = g_j^{q(t)}(x(t),u(t),w(t)), \] with the restriction that
algebraic equations are not allowed---that is, there may not be loops
in the dependency graph of algebraically-defined states.  In a given
discrete state $q$, a state variable must be defined either
algebraically or differentially.  However, the mode of definition for
a variable may change with the discrete state.

The discrete event evolution of a component is defined by a finite
state machine.  Edges in the state machine are labeled with {\em
guards}, {\em events} and {\em actions}.  A guard is a predicates on
the state of the component, its inputs, and the outputs of other
components.  A transition on an edge may be taken only when its guard
is true.  Events are points of synchronizations between components.
State machines in linked components synchronize according to {\em
event correspondence rules}, discussed in section~\ref{sync-rules}.  A
transition may trigger the execution of actions, which change the
state and output of the component, and create new components.

Associated to each discrete state is an {\em invariant}: a predicate
on the states of a component and the output of linked components,
which constrains their region of feasibility.  The behavior of the
system is undefined outside this region.

\subsection{World Model\label{world-model}}

The world is a directed graph of components, where the labeled edges are
{\em links}.  
The graph is encoded by the links $l(t)$ of components.
Let $a,\ b$ be components with $b \in l_a(t)$, then there is an
edge from $a$ to $b$.

The $l(t)$ evolves in time.  A component may change
$l(t)$ through {\em link} or {\em unlink} actions
associated with a transition.
